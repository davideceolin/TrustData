\section{Conclusions and Future Work}
\label{sec:conclusion}
We have explored various factors related to maintainability improvement effort, as preliminary work towards a more comprehensive prediction model for such effort. We have focused specifically on current values and change patterns of volume and maintainability. Starting from the hypotheses formulated in Section III, we have been able to establish that the possible amount of maintainability improvement can be predicted based on a linear model, depending on maintainability and volume of a system. Additionally we could establish a model quantifying the required effort for maintainability improvement in terms of daily code churn.

Apart from volume and maintainability, more factors need to be studied before an accurate prediction model for maintainability improvement effort can be obtained. These may include environmental factors (e.g. development method, developer skills, industry segment, etc) or additional technical factors (e.g. architectural styles, dependencies on third-party components, unit test quality). Also, we intend to make our analysis more detailed by distinguishing development phases (e.g. initial development versus maintenance), drilling down to the level of components or even files, and cross-linking with actual effort registrations.

%%% Local Variables:
%%% mode: latex
%%% TeX-master: "IWSM-Mensura-2016"
%%% End:
