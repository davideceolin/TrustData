\section{Problem Statement} 
\label{sec:ProblemStatement}
Following hypotheses could be leading towards a model for comparison-based maintainability improvement effort estimation and will be investigated in the following paragraphs:

\begin{description}
  \item[H1] Similarity-based sub-groups of systems show significantly different amount of maintainability change over the same time span.
  \item[H2] The observed amount of change correlates to the
current maintainability and system volume.
 \item[H3] The churn activity of strongly improving systems is higher than in hardly changing systems.
 \item[H4] The churn activity of strongly improving systems is significantly different from the activity in deteriorating systems.
\end{description}

Since the data is recorded in a context in which quality improvement is actively encouraged, we can assume that at least some of the observed systems in every subgroup were actively performing measures to improve their system's quality and that the ambitions were rather independent from the actual quality state of their system. We make thus the strong assumption that systems which improve in maintainability more than at least 90\% of all other systems can be regarded as the 'top-performer' which work strongly towards quality improvement. 
