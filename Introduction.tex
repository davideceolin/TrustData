\section{Introduction}
The maintainance of software is an important cost factor for organizations across all industries, making up approximately 40\% to 70\% of the total development costs of a software system~\cite{grubb2003software}.
Furthermore, research on the evolution of software has shown that recurring effort is needed to keep operational software functionally relevant and technically up-to-date~\cite{lehman1980programs}.  
Hence maintainability, the ``the degree of effectiveness and efficiency with which a product or system can be modified to improve it, correct it or adapt it to changes in environment, and in requirements''~\cite{iso25010}, is considered a key characteristic of software.

The {ISO 25010} definition of software quality characteristics, in particular maintainability, has given rise to measurement models that use properties of the source code comprising software systems.
One such measurement model, the SIG Maintainability Model~\cite{sig-maintainability-model}, which was first described by Heitlager et. al in 2007~\cite{heitlager2007practical}, is currently in operational use by the Software Improvement Group (SIG)\footnote{\url{http://www.sig.eu}}, a software consultancy based in Amsterdam, the Netherlands.
This model is capable of automatically determining a maintainability rating given a software system's source code and a (proprietary) benchmark database of previous measurements~\cite{alves2010deriving}.
The resulting maintainability ratings are expressed on a scale of 1 to 5 `stars', where a 3-star rating is calibrated to reflect a market-average maintainability.
The operational application of this model for a period of approximately 10 years has produced a significant set of data describing the evolution of maintainability of many industrial software systems.

This data set allows for the exploration of some questions regarding maintainability evolution, which we approach from the perspective of building a benchmark against which software systems can be compared.
In particular it is important to determine the base (change) rates of maintainability, i.e., the rates of change observed in the (benchmark) population prior to taking into account distinctive features, such as system size or age.

Knowing the base rates of maintainability evolution is necessary for the accurate interpretation of software measurement results~\cite{bruntink2015towards}.
Furthermore, rates of maintainability evolution are useful tools for estimation and planning of renovation and maintenance projects.
Most software systems will benefit from remaining at an already-high maintainability level, but what about systems that are currently poorly maintainable?
The question is when it would be feasible and beneficial to try and renovate such systems, and when it would be better to leave them as-is.

Concretely, in this paper we will discuss the following research questions:
\begin{itemize}
\item \textit{What is the base rate of maintainability change?}
\item \textit{Given that maintainability is known, do we observe different rates of change?}
\item \textit{What other features of software determine the rates of maintainability change?}
\item \textit{How could a benchmark of maintainability evolution be constructed?}
\end{itemize}

First, we provide some background information on the software metrics that were used in the research in Section~\ref{sec:background}. Then we describe the process of data collection in Section~\ref{sec:DataCollection}, followed by an exploratory data analysis in Section~\ref{sec:analysis}. Section~\ref{sec:Discussion} discusses the data analysis results from the perspective of the questions listed above. Related work is reflected in Section~\ref{sec:related}. Finally, Section~\ref{sec:conclusion} concludes and discussed opportunities for future work.

%%% Local Variables:
%%% mode: latex
%%% TeX-master: "IWSM-Mensura-2016"
%%% End:
