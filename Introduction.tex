\section{Introduction}
Software maintenance makes up approximately 40\% to 70\% of the total development costs of a software system \cite{grubb2003software}.
Maintainability of systems is thus an important cost factor and companies across all industries are interested in developing well maintainable software systems and in improving or maintaining the maintainability of existing systems.
SIG assists companies in this process by assessing the quality of their software systems, giving advice about possible improvements and monitoring of quality over time.
Maintainability is one of the main assessment criteria and is determined based on an ISO/IEC 25010 certified model using static code analysis \cite{heitlager2007practical}.

The respective organizations are interested in the current state, but also in the potential future development of their system.
\textit{How easy will it be to improve?} \textit{How likely is it to fall below a certain quality level?} Knowing future risks or potentials can help to utilize potentials and counteract negative trends in an early stage.
Experience of SIG's consultants with comparable projects and information about the system's current state helps them to answer those questions. 

However, manual assessment is time-consuming and the precise quantification of likelihoods is difficult.
Our aim is to support the assessment by quantifying the likelihood for improvement and deterioration, based on the history of comparable systems.
We will use a set of historic data from about 1500 software systems, group comparable systems and analyze how their quality changed over time.

The rest of this paper is structured as follows.
Section~\ref{sec:related} introduces related work.
Our hypotheses are introduced in Section~\ref{sec:ProblemStatement}, while our analyses are described in Section~\ref{sec:analysis}, and Section~\ref{sec:conclusion} concludes and outlines future work directions.
